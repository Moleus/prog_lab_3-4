\documentclass[a4paper, 11pt]{article}

\usepackage{forest} 

\usepackage{soul}
\usepackage{xcolor}
\usepackage{graphicx}
\usepackage{multicol}

\usepackage[hidelinks]{hyperref} 

% Poles settings:
\usepackage[left=30mm, top=10mm, right=20mm, bottom=10mm, head=5mm, foot=9mm]{geometry}

\usepackage{fontspec}
\setmainfont{FreeSerif}
\setsansfont{FreeSans}
\setmonofont{FreeMono}

% Polyglossia:
\usepackage{polyglossia}
\setdefaultlanguage{russian}
\newfontfamily\cyrillicfont{FreeSerif}[Script=Cyrillic]

% Source code blocks
\usepackage[skins,minted]{tcolorbox}
\definecolor{mintedbackground}{rgb}{1,1,1}
\definecolor{mintedframe}{rgb}{0.70,0.85,0.95}

% Filename extractor
\makeatletter
\DeclareRobustCommand{\filename}[1]{%
 \begingroup
  % \lstname seems to change hyphens into \textendash
  \def\textendash{-}%
  \filename@parse{#1}%
  \texttt{\filename@base.\filename@ext}%
 \endgroup
}
\makeatother

% Minted conifg
\setminted[java]{
    bgcolor=mintedbackground,
    fontfamily=tt,
    linenos=true,
    numberblanklines=true,
    numbersep=12pt,
    numbersep=5pt,
    gobble=0,
    frame=leftline,
    framesep=2mm,
    funcnamehighlighting=true,
    tabsize=2,
    obeytabs=false,
    mathescape=false
    samepage=false,
    showspaces=false,
    showtabs=false,
    texcl=false,
    baselinestretch=1.2,
    fontsize=\footnotesize,
    breaklines=true,
}

% One line code listing with filename extracted
\newtcbinputlisting[]{\javacode}[2][]{enhanced, listing engine=minted, 
listing only,#1,listing file={#2}, title=\filename{#2}, minted language=java, 
coltitle=mintedbackground!10!black, 
fonttitle=\ttfamily\small,
sharp corners, top=0mm, bottom=0mm,
title code={\path[draw=mintedframe,dashed](title.south west)--(title.south east);},
frame code={\path[draw=mintedframe](frame.south west) rectangle (frame.north east);}
}

\begin{document}
\Large
\thispagestyle{empty}
\begin{center}
Министерство науки и высшего образования Российской Федерации \\
Федеральное государственное автономное образовательное учреждение высшего образования \\
«Национальный исследовательский университет ИТМО» \\
\vspace{1em}
\textsl{Факультет Программной Инженерии и Компьютерной Техники}\\
\end{center}

\vspace{1em}

\thispagestyle{empty}
\begin{center}
\includegraphics[width=8cm]{imgs/itmo.jpg}
\end{center}

\vspace{3em}

\begin{center}
\large{
\textsc{\textbf{
Лабораторная работа №4 \linebreak 
% ООП \linebreak
Вариант 3617.1040}}
}
\end{center}

\vspace{12em}

\input{private.tex}

\newbox{\lbox}
\savebox{\lbox}{\hbox{\studName}}
\newlength{\maxl}
\setlength{\maxl}{\wd\lbox}
\hfill\parbox{11cm}{
\hspace*{5cm}\hspace*{-5cm}Студент:\hfill\hbox to\maxl{\studName\hfill}\\
\hspace*{5cm}\hspace*{-5cm}Преподаватель:\hfill\hbox to\maxl{\teacherName}\\
\\
\hspace*{5cm}\hspace*{-5cm}Группа:\hfill\hbox to\maxl{\groupNumber}\\
}

\vspace{\fill}

\begin{center}
Санкт-Петербург \\2021
\end{center}
\newpage

\tableofcontents
\vspace{2em}
\pagebreak{}

\section{Текст задания}
\definecolor{fg-pink}{RGB}{199, 37, 78}
\definecolor{bg-pink}{RGB}{249, 242, 244}
\sethlcolor{bg-pink}
\newcommand*{\codehl}[1]{\textcolor{fg-pink}{\hl{#1}}}
%
\textbf{Программа должна удовлетворять следующим требованиям:}
\begin{enumerate}
\item В программе должны быть реализованы 2 собственных класса исключений\\ (checked и unchecked), а также обработка исключений этих классов.
\item В программу необходимо добавить использование локальных, анонимных и вложенных классов (static и non-static).
\end{enumerate}

\textbf{Порядок выполнения работы:}
\begin{enumerate}
    \item  Доработать объектную модель приложения.
    \item  Перерисовать диаграмму классов в соответствии с внесёнными в модель изменениями.
    \item  Согласовать с преподавателем изменения, внесённые в модель.
    \item  Модифицировать программу в соответствии с внесёнными в модель изменениями.
\end{enumerate}

\textbf{Отчёт по работе должен содержать:}
\begin{enumerate}
\item  Текст задания.
\item  Диаграмма классов объектной модели.
\item  Исходный код программы.
\item  Результат работы программы.
\item  Выводы по работе.
\end{enumerate}
\pagebreak

\section{Описание предметной области для варианта 3617}
Подойдя к бывшему дому Совы, они застали там Всех-Всех-Всех, за исключением Иа. 
Кристофер Робин всем объяснял, что делать, и Кролик объяснял всем то же самое, на тот случай, если они не расслышали, и потом они все делали это. 
Они где-то раздобыли канат и вытаскивали стулья и картины, и всякие вещи из прежнего дома Совы, чтобы все было готово для переезда в новый дом. 
Кенга связывала узлы и покрикивала на Сову: "Я думаю, тебе не нужна эта старая грязная посудная тряпка.  Правда? 
И половик тоже не годится, он весь дырявый", на что Сова с негодованием отвечала: "Конечно, он годится-- надо только правильно расставить мебель! 
А это совсем не посудное полотенце, а моя шаль!" 
Крошка Ру поминутно то исчезал в доме, то появлялся оттуда верхом на очередном предмете, который поднимали канатом, что несколько нервировало Кенгу, потому что она не могла за ним как следует присматривать. 
Она даже накричала на Сову, заявив, что ее дом - это просто позор, там такая грязища, удивительно, что он не опрокинулся раньше! 
Вы только посмотрите, как зарос этот угол, просто ужас!  Там поганки! 
Сова удивилась и посмотрела, а потом саркастически засмеялась и объяснила, что это ее губка и что если уж не могут отличить самую обычную губку от поганок, то в хорошие времена мы живем!..


Robin,Rabbit {
	void describe(String message, String reason = null){
		System.out.println(something);
	}
}

All {
	void hear()
}

Kengu {
	void shoutAt(Entity entity, shoutedOnceFlag, message) {
		private verb;
		if (shoutedOnceFlag == True) {
			verb = "накричала";
		} else {
			verb = "покрикивала";
		}
		System.out.println(
				this.name + verb + " на " 
				+ entity.name 
				+ message
		)
	}
}

\pagebreak

\section{Диаграмма классов объектной модели}
% \intput{class_diagram.tex}

% \section{Исходный код программы}
% % \subsection{Основной класс}
% \small
% \javacode[]{../src/App.java}
% \subsection{Покемоны}
% \small
% \javacode[]{../src/pokemons/Magearna.java}
% \small
% \javacode[]{../src/pokemons/Burmy.java}
% \small
% \javacode[]{../src/pokemons/Wormadam.java}
% \small
% \javacode[]{../src/pokemons/Deino.java}
% \small
% \javacode[]{../src/pokemons/Zweilous.java}
% \small
% \javacode[]{../src/pokemons/Hydreigon.java}
% \subsection{Атаки}
% \small
% \javacode[]{../src/moves/RockTomb.java}
% \small
% \javacode[]{../src/moves/Swagger.java}
% \small
% \javacode[]{../src/moves/DizzyPunch.java}
% \small
% \javacode[]{../src/moves/Present.java}
% \small
% \javacode[]{../src/moves/Rest.java}
% \small
% \javacode[]{../src/moves/Facade.java}
% \small
% \javacode[]{../src/moves/PoisonPowder.java}
% \small
% \javacode[]{../src/moves/DoubleTeam.java}
% \small
% \javacode[]{../src/moves/Scald.java}
% \small
% \javacode[]{../src/moves/Confusion.java}

\section{Результат работы программы}
% \begin{center}
%   \inputminted[linenos=true, fontsize=\scriptsize]{text}{sample_output.txt}
% \end{center}

\section{Выводы по работе}
В ходе этой лабораторной работы я научился применять принципы SOLID на практике, подробнее разобрал интерфейсы, абстрактные классы и перечисления. Начал изучать систему сборки Maven.

\end{document}